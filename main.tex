\pdfminorversion=4
\documentclass[aspectratio=169]{beamer}

\mode<presentation>
{
  \usetheme{default}
  \usecolortheme{default}
  \usefonttheme{default}
  \setbeamertemplate{navigation symbols}{}
  \setbeamertemplate{caption}[numbered]
  \setbeamertemplate{footline}[frame number]  % or "page number"
  \setbeamercolor{frametitle}{fg=white}
  \setbeamercolor{footline}{fg=black}
} 

\usepackage[english]{babel}
\usepackage[utf8x]{inputenc}
\usepackage{tikz}
\usepackage{courier}
\usepackage{array}
\usepackage{bold-extra}
\usepackage{minted}
\usepackage[thicklines]{cancel}
\usepackage{fancyvrb}

\xdefinecolor{dianablue}{rgb}{0.18,0.24,0.31}
\xdefinecolor{darkblue}{rgb}{0.1,0.1,0.7}
\xdefinecolor{darkgreen}{rgb}{0,0.5,0}
\xdefinecolor{darkgrey}{rgb}{0.35,0.35,0.35}
\xdefinecolor{darkorange}{rgb}{0.8,0.5,0}
\xdefinecolor{darkred}{rgb}{0.7,0,0}
\definecolor{darkgreen}{rgb}{0,0.6,0}
\definecolor{mauve}{rgb}{0.58,0,0.82}

\title[2019-09-13-history-strangeloop]{The divergent histories of particle physics and computing}
\author{Jim Pivarski}
\institute{Princeton University -- IRIS-HEP}
\date{September 13, 2019}

\usetikzlibrary{shapes.callouts}

\begin{document}

\logo{\pgfputat{\pgfxy(0.11, 7.4)}{\pgfbox[right,base]{\tikz{\filldraw[fill=dianablue, draw=none] (0 cm, 0 cm) rectangle (50 cm, 1 cm);}\mbox{\hspace{-8 cm}\includegraphics[height=1 cm]{princeton-logo-long.png}\hspace{0.1 cm}\raisebox{0.1 cm}{\includegraphics[height=0.8 cm]{iris-hep-logo-long.png}}\hspace{0.1 cm}}}}}

\begin{frame}
  \titlepage
\end{frame}

\logo{\pgfputat{\pgfxy(0.11, 7.4)}{\pgfbox[right,base]{\tikz{\filldraw[fill=dianablue, draw=none] (0 cm, 0 cm) rectangle (50 cm, 1 cm);}\mbox{\hspace{-8 cm}\includegraphics[height=1 cm]{princeton-logo.png}\hspace{0.1 cm}\raisebox{0.1 cm}{\includegraphics[height=0.8 cm]{iris-hep-logo.png}}\hspace{0.1 cm}}}}}

% Uncomment these lines for an automatically generated outline.
%\begin{frame}{Outline}
%  \tableofcontents
%\end{frame}

% START START START START START START START START START START START START START

%% \begin{frame}{Prehistory}
%% \Large
%% \vspace{0.5 cm}
%% \begin{center}
%% We can start anywhere (Antikythera mechanism? Ishango bone?), but let's start with the Hollerith machine.
%% \end{center}
%% \end{frame}

%% \begin{frame}{The U.S.\ Census's problem}
%% \large
%% \vspace{0.35 cm}
%% The U.S.\ does a census every 10 years. The 1880 census took 8 years to process.
%% \begin{center}
%% $\longrightarrow$ Big data problem!
%% \end{center}

%% \vspace{0.15 cm}
%% Held a competition for a new method; winner was 10$\times$ faster than the rest:

%% \begin{center}
%% \includegraphics[height=5.3 cm]{hollerith.jpg}\hspace{1 cm}\includegraphics[height=5.3 cm]{1908_Hollerith_Machine.jpg}
%% \end{center}
%% \end{frame}

%% \begin{frame}{Census records on punch cards, which filtered electrical contacts}
%% \vspace{0.17 cm}
%% \begin{columns}
%% \column{0.5\linewidth}
%% \includegraphics[width=\linewidth]{1890_Census_Hollerith_Pantograph_Punching_Machine_Sci_Amer.jpg}

%% \column{0.5\linewidth}
%% \includegraphics[width=\linewidth]{1890_Hollerith_Circuit-Closing_Press_OM.jpg}
%% \end{columns}
%% \end{frame}

%% \begin{frame}{Wired to a machine that opens a door for each matching pattern}
%% \vspace{0.25 cm}
%% \begin{center}
%% \includegraphics[width=0.45\linewidth]{1890_Hollerith_Sorting_Machine_OM.jpg}
%% \end{center}
%% \end{frame}

%% \begin{frame}{It was an SQL machine: 3 basic clauses of most SQL queries}
%% \vspace{0.25 cm}
%% \begin{center}
%% \includegraphics[width=0.75\linewidth]{hh-tabulator.pdf}
%% \end{center}

%% \mbox{ } \hfill \mintinline{sql}{SELECT name WHERE literate GROUP BY marital_status} \hfill \mbox{ }
%% \end{frame}

%% \begin{frame}{Origin of business computing}
%% \Large
%% \vspace{0.5 cm}
%% Herman Hollerith founded a company selling these machines, which, after a series of mergers, became International Business Machines.

%% \small
%% \vspace{1 cm}
%% \textcolor{blue}{\url{https://www.officemuseum.com/data_processing_machines.htm}}

%% \vspace{0.25 cm}
%% \textcolor{blue}{\url{https://www.ibm.com/ibm/history/exhibits/builders/builders_hollerith.html}}
%% \end{frame}

%% \begin{frame}{Physics interest in computing came later}
%% \Large
%% \vspace{0.5 cm}
%% Nuclear/particle physics was a tabletop science before the Manhattan Project (nuclear bomb).

%% \normalsize
%% \vspace{0.75 cm}
%% \hfill\mbox{\includegraphics[height=3.3 cm]{bigsci-lblstaff.jpg}\hspace{-0.7 cm}}

%% \vspace{-3.8 cm}
%% \textcolor{darkblue}{\large Exceptions:}
%% \begin{itemize}
%% \item Ernest Lawrence's group at Berkeley (invented accelerators)

%% employed dozens---disparaged as ``Berkeleitis.''

%% \textcolor{blue}{\small\href{https://history.aip.org/history/exhibits/lawrence/epa.htm}{\tt https://history.aip.org/history/exhibits/\\lawrence/epa.htm}}

%% \item Cryogenics was big science in the early 20th century, but

%% didn't require much number-crunching.
%% \end{itemize}

%% \Large
%% \vspace{0.5 cm}
%% Physics and computing don't converge until the 1940's.
%% \end{frame}

%% \begin{frame}{\only<1>{Physicists got into computers when they became general-purpose}\only<2>{Computers got general-purpose when physicists got involved}}
%% \vspace{0.5 cm}

%% \begin{columns}
%% \column{0.3\linewidth}
%% \includegraphics[width=\linewidth]{presper-and-mauchly.jpg}

%% \column{0.7\linewidth}
%% 1944: John Mauchly (physicist) and J.\ Presper Eckert (electrical engineer) designed ENIAC to replace mechanical computers for ballistics.

%% \vspace{0.25 cm}
%% ENIAC was one of the first computers \textcolor{darkblue}{driven by machine code instructions}, stored as a program in memory.
%% \end{columns}

%% \vspace{0.25 cm}
%% \begin{columns}
%% \column{0.65\linewidth}
%% 1945: John von Neumann learned of their work and suggested using it for nuclear simulations (H-bomb).

%% \vspace{0.25 cm}
%% His internal memo describing ENIAC's stored programs was leaked; now known as ``Von Neumann architecture.''

%% \vspace{0.25 cm}
%% Los Alamos group led by Nicholas Metropolis, developed Monte Carlo techniques for physics problems.
%% \column{0.35\linewidth}
%% \includegraphics[width=\linewidth]{neumann_oppie.jpg}
%% \end{columns}
%% \end{frame}

%% \begin{frame}{Alternatives: the FERMIAC}


%% \begin{center}
%% \includegraphics[height=5.5 cm]{FERMIAC.jpg}\hspace{0.25 cm}\includegraphics[height=5.5 cm]{STAN_ULAM_HOLDING_THE_FERMIAC.jpg}
%% \end{center}

%% \small
%% \vspace{0.5 cm}
%% \textcolor{blue}{\url{https://www.tandfonline.com/doi/abs/10.1080/23324309.2018.1514312}}

%% \end{frame}

%% \begin{frame}{{\bf M}etropolis {\bf A}nd {\bf N}eumann {\bf I}nvent {\bf A}wful {\bf C}ontraption}
%% \small
%% \vspace{0.25 cm}

%% \includegraphics[height=5 cm]{metropolis.jpg}\hspace{0.1 cm}\includegraphics[height=5 cm]{1952-Maniac-1.jpg}

%% \vspace{0.25 cm}
%% \begin{columns}
%% \column{1.1\linewidth}
%% \textcolor{blue}{\url{https://www.atomicheritage.org/history/computing-and-manhattan-project}}

%% \vspace{0.1 cm}
%% \textcolor{blue}{\url{https://www.manhattanprojectvoices.org/oral-histories/nicholas-metropolis-interview}}

%% \vspace{0.1 cm}
%% \textcolor{blue}{\url{https://www.jstor.org/stable/20025423}}

%% \vspace{0.1 cm}
%% \textcolor{blue}{\url{https://books.google.com/books?id=qB819m2ibUQC}}
%% \end{columns}
%% \end{frame}

%% \begin{frame}{The actual programming was performed by these six women}
%% \vspace{-0.4 cm}
%% \begin{columns}[t]
%% \column{0.15\linewidth}
%% \begin{center}
%% \includegraphics[width=\linewidth]{Kay-McNulty.jpg}

%% Kathleen McNulty
%% \end{center}

%% \column{0.15\linewidth}
%% \begin{center}
%% \includegraphics[width=\linewidth]{Fran-Bilas.jpg}

%% Frances Bilas
%% \end{center}

%% \column{0.15\linewidth}
%% \begin{center}
%% \includegraphics[width=\linewidth]{Betty-Jennings.jpg}

%% Betty Jean Jennings
%% \end{center}

%% \column{0.15\linewidth}
%% \begin{center}
%% \includegraphics[width=\linewidth]{Ruth-Lichterman.jpg}

%% Ruth Lichterman
%% \end{center}

%% \column{0.15\linewidth}
%% \begin{center}
%% \includegraphics[width=\linewidth]{Betty-Snyder.jpg}

%% Elizabeth Snyder
%% \end{center}

%% \column{0.15\linewidth}
%% \begin{center}
%% \includegraphics[width=\linewidth]{Marlyn-Meltzer.jpg}

%% Marlyn Wescoff
%% \end{center}
%% \end{columns}

%% \scriptsize
%% \vspace{0.3 cm}
%% Kathy Kleiman's research:

%% \vspace{0.1 cm}
%% \textcolor{blue}{\url{http://eniacprogrammers.org/eniac-programmers-project/}}

%% \vspace{0.2 cm}
%% There are many secondary-source articles like this one:

%% \vspace{0.1 cm}
%% \textcolor{blue}{\url{http://mentalfloss.com/article/53160/meet-refrigerator-ladies-who-programmed-eniac}}

%% \vspace{0.2 cm}
%% This is a fantastic article; an overview of women in computing until the 1980's:

%% \vspace{0.1 cm}
%% \textcolor{blue}{\url{https://www.nytimes.com/2019/02/13/magazine/women-coding-computer-programming.html}}

%% \vspace{0.2 cm}
%% First assembly language invented by Kathleen Booth:

%% \vspace{0.1 cm}
%% \textcolor{blue}{\url{http://bobmackay.com/Booth/Booth.html}}
%% \textcolor{blue}{\url{http://www.computinghistory.org.uk/det/32489/Kathleen-Booth/}}
%% \end{frame}

%% \begin{frame}[fragile]{Eckert-Mauchly Computer Corporation $\to$ Remington Rand}
%% \vspace{0.5 cm}

%% \begin{columns}
%% \column{0.7\linewidth}
%% Mauchly and Eckert ``went into industry'' selling computers; the first one (UNIVAC) to the U.S.\ Census.

%% \vspace{0.5 cm}
%% 1950: Short Code, the first executable high-level language: \\ a transliterated interpreter of mathematical formulas.
%% \begin{center}
%% \small
%% \vspace{-0.1 cm}
%% \begin{minipage}{0.8\linewidth}
%% \begin{verbatim}
%% math: X3 =  (  X1 +  Y1 )  /  X1 * Y1
%% code: X3 03 09 X1 07 Y1 02 04 X1   Y1
%% \end{verbatim}
%% \end{minipage}

%% \vspace{0.2 cm}
%% \normalsize
%% 50$\times$ slower than machine code because it was interpreted.
%% \end{center}

%% \vspace{0.25 cm}
%% 1952--1959: At Remington Rand, Grace Hopper developed a series of {\it compiled} languages, ultimately COBOL.

%% \vspace{0.25 cm}
%% Meanwhile, IBM developed FORTRAN: 1954--1957.

%% \small
%% \vspace{0.25 cm}
%% \textcolor{blue}{\url{http://www.historyofinformation.com/detail.php?entryid=839}}

%% \column{0.3\linewidth}
%% \includegraphics[width=\linewidth]{Univac_I_at_Census_Bureau_with_two_operators.jpg}

%% \includegraphics[width=\linewidth]{Grace_Hopper_and_UNIVAC.jpg}
%% \end{columns}
%% \end{frame}

%% \begin{frame}{The physics analysis problem}
%% \large
%% \vspace{0.35 cm}

%% Particle collision/decay images are rich datasets. Analyzers need to infer ``who decayed to whom'' and measure the curvature of tracks, which yields momentum.

%% \begin{center}
%% \includegraphics[height=6 cm]{omega-minus-1.png}\hspace{0.2 cm}\includegraphics[height=6 cm]{omega-minus-2.png}
%% \end{center}
%% \end{frame}

%% \begin{frame}{Fully manual $\to$ human/computer $\to$ fully automated}
%% \vspace{0.35 cm}
%% \begin{columns}[b]
%% \column{0.55\linewidth}
%% \begin{columns}[b]
%% \column{0.4\linewidth}
%% \includegraphics[height=2.9 cm]{franckenstein-2.jpg}

%% \column{0.47\linewidth}
%% \$2M bubble chamber, \$0.2M IBM 650

%% \vspace{0.2 cm}
%% Jack Franck's ``Franckenstein'' converted positions marked along tracks
%% \end{columns}

%% into numbers on punch cards for analysis.

%% \vspace{0.2 cm}
%% \scriptsize
%% \textcolor{blue}{\url{https://www2.lbl.gov/Science-Articles/Research-Review/Magazine/1981/81fchp6.html}}

%% \vspace{0.2 cm}
%% \includegraphics[height=2.9 cm]{franckenstein-1.jpg}\hfill\includegraphics[height=2.9 cm]{franckenstein-4.jpg}

%% \column{0.55\linewidth}
%% Madeleine Isenberg describes being a ``scanner.''

%% \scriptsize
%% \vspace{0.1 cm}
%% \textcolor{blue}{\url{http://www.physics.ucla.edu/marty/HighEnergyPhysics.pdf}}

%% \vspace{0.2 cm}
%% \includegraphics[width=\linewidth]{franckenstein-3.jpg}
%% \end{columns}
%% \end{frame}

%% \begin{frame}{Processing events thousands of times faster}
%% \vspace{0.35 cm}
%% \begin{columns}
%% \column{0.6\linewidth}
%% 1959 paper on the software infrastructure: {\tt PANG}, {\tt KICK}, {\tt EXAMIN}, {\tt DRIVEL}

%% \small
%% \vspace{0.2 cm}
%% \textcolor{blue}{\url{http://inspirehep.net/record/919917/files/HEACC59_575-583.pdf}}

%% \normalsize
%% \vspace{0.4 cm}
%% {\it Image and Logic: A Material Culture of Microphysics} by Peter Galison:

%% \small
%% \vspace{0.2 cm}
%% \textcolor{blue}{\url{https://books.google.com/books?id=6Gcu92U8CwYC\&lpg=PA373\&ots=8YV5cPePE6\&dq=Franckenstein\%20bubble\&pg=PA373\#v=onepage\&q=Franckenstein\&f=false}}

%% \normalsize
%% \vspace{0.4 cm}
%% {\it Discovering Alvarez: Selected Works of Luis W. Alvarez}: that plot $\longrightarrow$

%% \column{0.45\linewidth}
%% \includegraphics[width=\linewidth]{scaleup.png}
%% \end{columns}
%% \end{frame}

%% \begin{frame}{Getting more automated: the spiral reader}
%% \vspace{0.35 cm}
%% Known as the ``LSD'' (Lecteur \`a Spirale Digitis\'ee) at CERN. The analyst only needs to center the spiral on the vertex, specify the number of tracks, and {\tt POOH} finds them.

%% \small
%% \vspace{0.2 cm}
%% \textcolor{blue}{\url{https://escholarship.org/content/qt7hf5r27c/qt7hf5r27c.pdf}}

%% \vspace{0.5 cm}
%% \includegraphics[height=3 cm]{spiral-reader-2.jpg}\hspace{0.2 cm}\includegraphics[height=4.5 cm]{spiral-reader-1.jpg}\hspace{0.2 cm}\includegraphics[height=4.5 cm]{spiral-reader-3.jpg}

%% \end{frame}

%% \begin{frame}{Detectors gets digitized; tracking algorithms get automated}
%% \vspace{0.25 cm}
%% \begin{columns}
%% \column{0.6\linewidth}
%% \includegraphics[width=\linewidth]{cms-tracker.jpg}

%% \vspace{-4 cm}
%% \includegraphics[height=4 cm]{track-reconstruction.png}

%% \column{0.35\linewidth}
%% \includegraphics[width=\linewidth]{megatek-2.png}

%% \includegraphics[width=\linewidth]{cms-event.png}
%% \end{columns}
%% \end{frame}

%% \begin{frame}{CERN Courier special issues on computing}
%% \large
%% \begin{columns}
%% \column{0.55\linewidth}
%% \begin{center}
%% September 1967

%% \vspace{0.2 cm}
%% \includegraphics[width=0.55\linewidth]{cern-courier-1.png}

%% \vspace{0.2 cm}
%% \scriptsize
%% \textcolor{blue}{\url{https://cds.cern.ch/record/1728900?ln=en}}
%% \end{center}

%% \column{0.55\linewidth}
%% \begin{center}
%% March 1972

%% \vspace{0.2 cm}
%% \includegraphics[width=0.55\linewidth]{cern-courier-2.png}

%% \vspace{0.2 cm}
%% \scriptsize
%% \textcolor{blue}{\url{https://cds.cern.ch/record/1729464?ln=en}}
%% \end{center}
%% \end{columns}
%% \end{frame}

\begin{frame}{Divergence}
\includegraphics[width=\linewidth]{hydra-1.png}
\end{frame}



\end{document}
